\documentclass[./]{subfiles}

%Bernstein Polynomials
\begin{document}
Bernstein polynomials are defined over $\xi \in [0,1]$ as 
\begin{alignat}{3}
B_{a, p}(\xi) &= (1-\xi) B_{a, p-1}(\xi) + \xi B_{a-1,p-1}(\xi), \\
B_{1, 0}(\xi) &= 1, \\
B_{a, p}(\xi) &= 0 \text{ if } a<1 \text{ or } a > p+1.
\end{alignat}

We will use Bernstein polynomials as the local description of elements over our surface, and relate them to the global basis functions via the process of \textit{B\`ezier Extraction}. These polynomials have several handy properties with regard to their implementation as a basis for analysis. First, they are interpolatory at the endpoints of the domain $[0,1]$, and are a partition of unity, i.e.:
\begin{equation*}
\sum_{a=1}^{p+1} B_{a,p}(\xi) = 1.
\end{equation*}
\end{document}